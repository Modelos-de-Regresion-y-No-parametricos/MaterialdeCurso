
\newcommand{\ind}{\perp}

\newcommand{\re}{\mathbb{R}}
\newcommand{\prob}[1]{\mathbb P \! \left[ #1 \right]}
\newcommand{\probc}[2]{\mathbb P \! \left[ #1 \, | #2 \right]}
\newcommand{\esp}[1]{ \mathbb E \! \left[ #1 \right] }
\newcommand{\espc}[2]{\mathbb E \! \left[ #1 | #2 \right]}
%\newcommand{\espc}[2]{\ensuremath{\imf{\se}{\cond{#1}{#2}}}}

\newcommand{\proint}[2]{\langle #1,#2\rangle}

% Alan commands

\newcommand{\indic}{\mathbbm{1}}

\renewcommand{\d}{\mathrm d}

\newcommand\restr[2]{{% we make the whole thing an ordinary symbol
  \left.\kern-\nulldelimiterspace % automatically resize the bar with \right
  #1 % the function
  \vphantom{\big|} % pretend it's a little taller at normal size
  \right|_{#2} % this is the delimiter
  }}
  
\newcommand\eval[3]{{% we make the whole thing an ordinary symbol
  \left.\kern-\nulldelimiterspace % automatically resize the bar with \right
  #1 % the function
  \vphantom{\big|} % pretend it's a little taller at normal size
  \right|_{#2}^{#3} % this is the delimiter
  }}
